\documentclass[main]{subfiles}
\begin{document}

\section{Lab Questions and Answers}
This is the collection of ”What we expect you to know” sections from the individual labs. If you can tell someone else with confidence about these topics, you will do well at the exam. We recommend that you try to study with another student. Take turns explaining to each other about these topics, and drawing schematics and diagrams for each other (This was written by Tobi Delbrück et al. to the students as a study sheet. We extracted the questions that can be properly answered as they were treated in the lecture and labs. The original document can be found on the AVLSI webpage at INI).
\subsection{Lab 1}
\begin{itemize}
\item What is neuromorphic aVLSI all about? 
\item Who came up with it?
\item Where is it being done?
\item Semiconductors, bands, band gap of silicon, the Fermi-Dirac distribution, donors and acceptors, PN junction, reverse and forward characteristics.
\item Using the Keithley voltage source (K230) and source-measure electrometer (K236) instruments. E.g. what is the K236 DC input impedance when measuring voltage with it? 
\item Basic knowledge of matlab. How to use the pot-boxes.
\item Remember: Experimental techniques and devices are never ideal.
\end{itemize}

\subsection{Lab 2}
\begin{itemize}
\item What does it mean for a MOS transistor channel to be accumulated, flat-band, depleted, inverted? 
\item Knowledge of how subthreshold transistor operation is a diffusion process and why it depends exponentially on the terminal voltages. 
\item What is the meaning of ”saturation”?
\item What is the triode or linear operating range? $I_{ds}$ vs $V_{gs}$ on log scale. 
\item Differences between n- and p-fets. 
\item Typical values of $I_0$, $\kappa$ and subthreshold operating range. 
\item What are wells and how should the wells be biased relative to the substrate? 
\item What is the ”back gate” or ”body effect”? 
\item How is the back gate is related to $\kappa$? 
\item How do you measure $\kappa$? 
\item How to make a MOS capacitor and what is its C-V relationship. 
\item How a source follower works and how to compute the gain of a source follower.
\end{itemize}

\subsection{Lab 3}
\begin{itemize}
\item How transistors work above threshold. 
\item What is the linear or triode region and what is the saturation region? 
\item How they depend on gate and threshold voltage. What is the “overdrive”? 
\item What is the specific current? How the Early effect comes about. 
\item Typical values for Early voltage. 
\item How to sketch graphs of transistor current vs. gate voltage and drain-source voltage. 
\item How above-threshold transistors go into saturation and why the saturation voltage is equal to the gate overdrive. 
\item The above-threshold current equations. 
\item How above-threshold current depends on Cox and mobility.
\item How transconductance and drain resistance combine to generate voltage gain and what is the intrinsic voltage gain of a transistor.
\item What effect does velocity saturation have on transistor operation, specifically, how does it change the relation between saturation current and gate voltage? 
\item What is DIBL (drain induced barrier lowering) and II (impact ionization)?
\item What is the dominant source of mismatch? 
\item How does transistor mismatch scale with transistor size? 
\item What are typical values of transistor threshold voltage mismatch?
\item How do transconductance and drain resistance combine to generate voltage gain?
\end{itemize}



\subsection{Lab 4}
\begin{itemize}
\item Can you sketch a transamp, a wide range transamp, a current correlator, and a bump circuit in both n- and p-type varieties?
\item How does a differential pair work? 
\item How does the common-node voltage change with the input voltages? 
\item How can you compute the differential tail currents from the subthreshold equations, and how do you obtain the result in terms of the differential input voltage?
\item How does a current-correlator work? 
\item How does a bump circuit work?
\item The I-V characteristics of a transconductance amplifier below threshold. 
\item What’s the functional difference between simple and wide-output-range transamp? 
\item The subthreshold transconductance $g_m$ . 
\item The relation between gain A, transistor drain conductances $g_d$ , and transconductances $g_m$ .
\item Can you reason through all the node voltages in these circuits? I.e., if we draw the circuit and provide specific power supply and input voltages, can you reason to estimate all the other node voltages, at least to first order approximations, assuming $\kappa$ = 1?
\end{itemize}



\subsection{Lab 6}
\begin{itemize}
\item How does the WTA circuit work? 
\item Can you reason through its behavior? 
\item How does the bias current affect its performance? 
\item How can you adjust the gain of the circuit through the sizing of the transistors?
\item How to use a non-linear circuit, such as a operational amplifier, to compute
linear functions.
\item What a unity-gain follower is used for.
\item How to build a linear current-to-voltage converter (although we didn’t build one in the lab).
\end{itemize}


\subsection{Lab 7}
\begin{itemize}
\item How to use an oscilloscope and a function generator. 
\item How to compute the time-constant of a low-pass filter and how to estimate it from the measurements. 
\item How to change the time-constant of a follower-integrator circuit. 
\item In what way does the follower integrator behave non-linearly for large signal input?
\item The idea of using a lowpass filter (an integrator) to make a highpass
filter (a differentiator).
\item How the differentiator circuit is not a real differentiator but only an approximation over some frequencies defined by the time constant. 
\item How to sketch the transfer function of a differentiator circuit, showing the time constant on the sketch. 
\item How to implement a simple follower-differentiator, and a hysteretic differentiator, using followers. 
\item How to estimate the time-constants of both type of differentiators and how to
estimate them from the measurements.
\item How these circuits behave when driven with large signals. 
\item What is a canonical second-order system, and how to compute stability.
\end{itemize}


\subsection{Lab 8}
\begin{itemize}
\item How does the I-V curve of a diode change in the presence of light? 
\item How does phototransduction occur in silicon?
\end{itemize}


\subsection{Lab 9}
\begin{itemize}
\item How you can use adaptation in a feedback loop to high-pass amplify signal but not mismatch.
\item How you can build a fast logarithmic current-sense amplifier, by using feedback to make a virtual ground. 
\item How you can use a capacitive divider in the feedback loop of an amplifier to set a gain. 
\item How you can use a cascode configuration to increase effective drain resistance.
\item What is the Miller effect?
\item How a cascode can be used to nullify it?
\end{itemize}


\subsection{Lab 10}
\begin{itemize}
\item The schematic for a synapse circuit. 
\item How the synaptic current changes as a function of the synaptic weight, the time constant, and the presynaptic frequency.
\end{itemize}


\subsection{Lab 11}
\begin{itemize}
\item What is a neuron and what are its components (synapse, soma, dendrite)? 
\item What types of models are used to simulate neurons? 
\item How does the spike-generating mechanism work?
\item What is an FI curve? 
\item Can you draw the circuit schematic of the axon-hillock neuron?
\end{itemize}


\subsection{Lab 12}
%Joachim
\begin{itemize}
\item How do tunnelling and injection mechanisms work? 
\item What is the shape of the energy band diagram in the channel and oxide during tunnelling and injection? 
\item How are the memory cell circuits used to control tunnelling and injection?
\end{itemize}
\end{document}