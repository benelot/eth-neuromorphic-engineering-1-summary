\documentclass[report]{subfiles}
\begin{document}

\section{Lab Questions}
\subsection{Lab 1}
\begin{itemize}
\item Remember: Experimental techniques and devices are never ideal.
\end{itemize}

\subsection{Lab 2}
\begin{itemize}
\item What does it mean for a MOS transistor channel to be accumulated, flat-band, depleted,
inverted? \item Knowledge of how subthreshold transistor operation is a diffusion process and
why it depends exponentially on the terminal voltages. \item What is the meaning of ”saturation”?
\item What is the triode or linear operating range? $I_{ds}$ vs $V_{gs}$ on log scale. \item Differences between
n- and p-fets. \item Typical values of $I_0$, $\kappa$ and subthreshold operating range. \item What are wells and how should the wells be biased relative to the substrate? \item What is the ”back gate” or ”body
effect”? \item How is the back gate is related to $\kappa$? \item How do you measure $\kappa$? \item How to make a
MOS capacitor and what is its C-V relationship. \item How a source follower works and how to
compute the gain of a source follower.
\end{itemize}

\subsection{Lab 3}
\begin{itemize}
\item How transistors work above threshold. \item What is the linear or triode region and what is the
saturation region? \item How they depend on gate and threshold voltage. What is the “over-
drive”? \item What is the specific current? How the Early effect comes about. \item Typical values for
Early voltage. \item How to sketch graphs of transistor current vs. gate voltage and drain-source
voltage. \item How above-threshold transistors go into saturation and why the saturation voltage
is equal to the gate overdrive. \item The above-threshold current equations. \item How above-threshold
current depends on Cox and mobility.
\item How transconductance and drain resistance combine to generate voltage gain and what is
the intrinsic voltage gain of a transistor.
\item What effect does velocity saturation have on transistor operation, specifically, how does
it change the relation between saturation current and gate voltage? \item What is DIBL (drain
induced barrier lowering) and II (impact ionization)?
\item What is the dominant source of mismatch? \item How does transistor mismatch scale with tran-
sistor size? \item What are typical values of transistor threshold voltage mismatch?
\end{itemize}



\subsection{Lab 4}
\begin{itemize}
\item Can you sketch a transamp, a wide range transamp, a current correlator, and a bump circuit
in both n- and p-type varieties?
\item How does a differential pair work? \item How does the common-node voltage change with the
input voltages? \item How can you compute the differential tail currents from the subthreshold
equations, and how do you obtain the result in terms of the differential input voltage?
\item How does a current-correlator work? \item How does a bump circuit work?
\item The I-V characteristics of a transconductance amplifier below threshold. \item What’s the
functional difference between simple and wide-output-range transamp? \item The subthreshold
transconductance gm . \item The relation between gain A, transistor drain conductances gd , and
transconductances gm .
\item Can you reason through all the node voltages in these circuits? I.e., if we draw the circuit
and provide specific power supply and input voltages, can you reason to estimate all the
other node voltages, at least to first order approximations, assuming $\kappa$ = 1?
\end{itemize}



\subsection{Lab 6}
\begin{itemize}
\item How does the WTA circuit work? \item Can you reason through its behavior? \item How does the bias
current affect its performance? \item How can you adjust the gain of the circuit through the sizing
of the transistors?
\end{itemize}


\subsection{Lab 7}
\begin{itemize}
\item How to use an oscilloscope and a function generator. \item How to compute the time-constant of
a low-pass filter and how to estimate it from the measurements. \item  to change the time-
constant of a follower-integrator circuit. \item In what way does the follower integrator behave
nonlinearly for large signal input?
\end{itemize}


\subsection{Lab 8}
\begin{itemize}
\item How does the I-V curve of a diode change in the presence of light? \item How does phototrans-
duction occur in silicon?
\end{itemize}


\subsection{Lab 9}
\begin{itemize}
\item How you can use adaptation in a feedback loop to highpass amplify signal but not mismatch.
\item How you can build a fast logarithmic current-sense amplifier, by using feedback to make a
virtual ground. \item How you can use a capacitive divider in the feedback loop of an amplifier to
set a gain. \item How you can use a cascode configuration to increase effective drain resistance.
\item What is the Miller effect, and how a cascode can be used to nullify it.
\end{itemize}


\subsection{Lab 10}
\begin{itemize}
\item The schematic for a synapse circuit. \item How the synaptic current changes as a function of the
synaptic weight, the time constant, and the presynaptic frequency.Can
\end{itemize}


\subsection{Lab 11}
\begin{itemize}
\item What is a neuron and what are its components (synapse, soma, dendrite)? \item What types of
models are used to simulate neurons? \item How does the spike-generating mechanism work?
\item What is an FI curve? \item Can you draw the circuit schematic of the axon-hillock neuron?
\end{itemize}


\subsection{Lab 12}
%Joachim
\begin{itemize}
\item How do tunneling and injection mechanisms work? \item What is the shape of the energy band
diagram in the channel and oxide during tunneling and injection? \item How are the memory cell
circuits used to control tunneling and injection?
\end{itemize}
\end{document}